\chapter{Introducción}

\section{Objetivo}
Trazar la curva de contrastación de un instrumento a los fines de eliminar el error sistemático del mismo y verificar el indice de clase del instrumento contrastado

\section{Materiales e Instrumental necesarios}
\begin{itemize}
  \item Voltímetro digital (para ser empleado como patron).
    \footnote{Se deberá emplear como patron un Multímetro digital de buena calidad, si es posible con visor de 4 \sfrac{1}{2} cifras, con su correspondiente manual de especificaciones.}
  \item Voltímetro analógico (puede ser un Multímetro), a contrastar.
  \item Fuente de alimentación de $0V$ a $30V$.
\end{itemize}

\section{Introducción}
En este trabajo práctico se obtendrá la curva de contrastación de un Voltímetro de CC.

Simultáneamente se podrá además verificar la Clase del mismo (que es la forma en la cual se especifica la exactitud de estos instrumentos).

Se usará para ello como patrón un Voltímetro digital, pues se supone que estos, en su mayoría, presentan características de exactitud que superan ampliamente a la mayoría de los instrumentos analógicos.

No obstante habrá que consultar la hoja de especificaciones del mismo para estar seguro, y para este trabajo práctico se considerará satisfecha esta condición si su exactitud por lo menos cinco veces mejor que la esperada en el instrumento bajo pruebas.
