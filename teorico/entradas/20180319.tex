\chapter*{Introduccion a la Materia}
\section*{Bibliografia}
Instrumentos electronicos basicos (un libro un tanto viejo)
Mediciones para estudiantes de ingeniera, apunte en Autogestion

\section*{Modalidad}
11 guias de trabajos practicos en Autogestion
Se aprueba con 6 (60\%?)

\chapter{Errores e Incertidumbre}

\begin{description}
  \item[Error Absoluto:] \hfill \\ La diferencia entre el valor medido y el valor verdadero
    $X_m - X_v = \Delta_x$
  \item[Error Relativo:] \hfill \\ El cociente entre el error absoluto y el valor verdadero
    $\frac{\Delta_x}{X_v} = e$
  \item[Error porcentual:] \hfill \\ El error relativo expresado como porcentajes
    $\frac{\Delta_x}{X_v} \cdot 100 = e\%$
\end{description}

El error se determina con un patron, que es el que nos da la medida 
del valor verdadero; el patron deberia ser aproximadamente 5 veces
mas preciso que el instrumento a ensayar

\begin{description}
  \item[Incertidumbre:] \hfill \\ Se determina mediante las especificaciones de exactitud
    del instrumento $$\pm \Delta_x$$
    Como no conozco el valor verdadero uso el valo medido para saber el error. Es muy
    aproximado al verdadero $\pm e = \pm \frac{\Delta_x}{X_m}$
    $\pm e\% = \pm \frac{\Delta_x}{X_m}\cdot 100$
  \item[Exactitud:] \hfill \\ Es cuando el aparato de medida toma exactamente el valor real
  \item[Presicion:] \hfill \\ Es cuando una medicion repetida da siempre el mismo valor
  \item[Sensibilidad:] \hfill \\ Es la relacion entre la excitacion del instrumento y su
    respuesta $\frac{Resp}{Excitacion}$
    \textbf{TUDU HACER EL GRAFICO DE LOS DISPLAYS ANALOGICOS}
    En este caso, el display B es mas sensible, por lo que B debe tener otro fondo de escala,
    valor maximo o alcance, son equivalentes, aunque no se traduce muy bien a medidores digitales.
  \item[Resolucion:] \hfill \\ Por ejemplo un multimetro de $3  1/2$ cifras. La lectura
    maxima es 1999, tiene 2000 cuentas, de 0 a 1999.
    \textbf{TUDU HACER EL GRAFICO DE LOS DISPLAYS}
    Las mediciones seran las mismas pero disponemos de mas resolucion
\end{description}
