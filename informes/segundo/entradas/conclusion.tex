\chapter{Conclusion}
\section{Conclusiones}
Elabore conclusiones teniendo en cuenta lo siguiente:
\begin{itemize}
  \item ¿Que ventajas presenta el método de medición con 4 terminales  por sobre los métodos convencionales?
    \begin{itemize}
      \item La medición de 4 terminales es que permite medir valores pequeños de resistencia, sin importar la resistencia de los terminales del multímetro.

        También, dependiendo del multímetro, puede significar un error mucho mas pequeño al medir tension y corriente, que si solo midiésemos resistencia, en nuestro caso, usando el UNI-T UT61C y el ProsKit MT5211, logramos una precision de 1.8\% en la resistencia, y si hubiésemos usado cualquiera de esos dos multímetros, la incertidumbre hubiese sido de (1\%+2dígitos[0.1$\Omega$]) y (0.8\%+5dígitos[0.1$\Omega$]), respectivamente, lo cual para valores altos de resistencias es mejor, pero a valores bajos, la suma de los dígitos significa mucho error porcentual.
    \end{itemize}
  \item Realice una búsqueda en Internet  sobre el tema “caja de resistencias de precisión”. Busque información sobre instrumentos de laboratorio que se emplean para la medición de resistencia. Examine las hojas de especificaciones de los  manuales  de cada uno de ellos e interprete las que corresponden a incertidumbre en la medición.
    \begin{itemize}
      \item Respuesta
    \end{itemize}
  \item La fuente de corriente propuesta para  este práctico, es muy simple pero bastante efectiva. Busque la hoja de datos del Circuito integrado LM317 e investigue como funciona.
    \begin{itemize}
      \item Acotación?
    \end{itemize}
  \item La ecuación que se emplea para el cálculo de la incertidumbre supone que los errores  parciales se suman. Algunos autores  sugieren que en algunas situaciones, una mejor estimación  de la incertidumbre se obtiene  efectuando la raíz cuadrada de la suma de los cuadrados de los errores parciales. Discuta con sus compañeros cual es la mejor opción en el caso de la experiencia que acaba de realizar.  
    \begin{itemize}
      \item En nuestro caso analizamos esto mientras desarrollábamos el practico, y optamos por utilizar dicha ecuación de la media cuadrática; ya que consideramos ambas mediciones totalmente independientes, al haber sido medidas con instrumentos totalmente distintos, en momentos y lugares distintos.
    \end{itemize}
\end{itemize}
