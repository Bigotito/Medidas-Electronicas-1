\chapter{Desarrollo}
\section{Procedimiento}
\begin{itemize}
  \item Montar el siguiente circuito (La resistencia es opcional por si la fuente de alimentación disponible no llegase a $0V$)
    
    \begin{center}\begin{circuitikz}[scale = 1.0, transform shape]
      \draw
      (0,2) to[V = $V_{cc}$] (0,0)
      (0,2) -- (2,2)
      (0,0) -- (2,0)
      (2,2) to[pR = $R_{(Opcional)}$, n=pote] (2,0)
      (pote.wiper) -- ++(0,1)
      (2,2) -- (5,2)
      (2,0) -- (5,0)
      (5,2) to[voltmeter, color=white, l = Patron, n=at1] (5,0)
      (5,2) -- (8,2)
      (5,0) -- (8,0)
      (8,2) to[voltmeter, color=white, l = Contrastado, n=vt1] (8,0)
      ;
      \rotarTester{at1}{0}
      \rotarTester{vt1}{0}
    \end{circuitikz}\end{center}
  \item Ajustar la tension de la fuente para lograr que la aguja indicadora del instrumento a contrastar se situé sobre cada una de las marcas correspondientes a las divisiones de su escala y simultáneamente tomar nota del valor indicado por el instrumento empleado como patron.
  \item Efectuar una pasada hacia arriba (Es decir, desde cero hasta el valor máximo) y una pasada hacia abajo (desde el valor máximo hacia el cero)
  \item Calcular los dos valores de error absoluto del instrumento (uno de la pasada hacia arriba y otro hacia abajo) en cada punto contrastado. Si no son iguales, consignar el mayor.
  \item Completar la siguiente tabla con los valores anteriores.
    \begin{description}
      \item $V_L$: Valores indicados del instrumento a contrastar.
      \item $V_p$: Valor medido con el instrumento patron
      \item $\Delta V$: Error absoluto del instrumento contrastado
    \end{description}

\end{itemize}
\begin{center}\begin{tabu}{|c|c|c|c|c|c|c|c|c|c|c|} \hline
  $V_L$                       & 1     & 2     & 3     & 4     & 5     & 6    & 7    & 8    & 9    & 10 \\ \hline
  $V_p$ (pasada hacia arriba) & 0.986 & 1.968 & 2.972 & 3.986 & 5.023 & 6.05 & 7.08 & 8.09 & 9.11 & 10.10 \\ \hline
  $V_p$ (pasada hacia abajo)  & 1.023 & 1.964 & 3.000 & 4.003 & 5.047 & 6.05 & 7.06 & 8.09 & 9.12 & 10.10\\ \hline
  $\Delta V$(el mayor valor)  & 0.023 & 0.036 & 0.028 & 0.014 & 0.047 & 0.05 & 0.08 & 0.09 & 0.12 & 0.10\\ \hline
\end{tabu}\end{center}

\begin{itemize}
  \item En base a los datos tabulados confeccionar la gráfica de corrección.
\end{itemize}

\begin{center}\tabulinesep=1.2mm\begin{tabu}{c c|c|c|c|}
  \cline{2-5}
  \multicolumn{1}{c|}{}            & Instrumento Contrastado                & Patron                                  & Fecha                       & Operador/es \\ \cline{2-5}
  \multicolumn{1}{c|}{}            & Marca: Ganz                            & Marca: UNI-T                            & \multirow{2}{*}{04/04/2018} & Della Santina Lucas \\
  \multicolumn{1}{c|}{Sumar ($V$)} & N\textsuperscript{\underline{o}}: 8032 & N\textsuperscript{\underline{o}}: UT61C &                             & Gratton Antonino\\ \cline{2-5}
  \multicolumn{5}{r}{
\begin{tikzpicture}\begin{axis}[
    xmin = 0, xmax= 11,
    ymin = -0.15, ymax = 0.15,
    xtick={0,1,2,3,4,5,6,7,8,9,10},
    legend pos=north west,
    grid=major,
    grid style=dashed,
    x post scale = 2
]
\addplot[
    ycomb,
    color=black,
    mark=*,
    ]
    coordinates {
      (1, -0.023)
      (2,  0.036)
      (3,  0.028)
      (4,  0.014)
      (5, -0.047)
      (6, -0.05)
      (7, -0.08)
      (8, -0.09)
      (9, -0.12)
      (10,-0.10)
    };
\end{axis}\end{tikzpicture}
  }\\
  Restar($V$)
\end{tabu}\end{center}

\begin{itemize}
  \item De los datos recogidos, identifique cual es el ''Error absoluto máximo'' y con ese valor calcule el indice de clase usando la siguiente expresión \footnote{Redondeando hacia arriba el resultado}:
\end{itemize}

$\text{Clase} = \dfrac{\text{Error absoluto maximo}}{\text{Valor de fondo de escala}} \cdot 100 \rightarrow \text{Clase} = \dfrac{\left|-0.12\right|}{10} \rightarrow \text{Clase} = 1.2$

$$\framebox{\text{Clase del Instrumento} = 1.5}$$

\begin{itemize}
  \item Compruebe si la ''Clase'' obtenida guarda coherencia con la indicada en la hoja de especificaciones del voltímetro contrastado.
\end{itemize}

La Clase del instrumento según sus especificaciones es de 2.5, y según el máximo error que obtuvimos, su clase es de 1.5.

Esto es coherente, ya que la especificación de clase del instrumento tiene en cuenta el mayor error posible, y resulta que en nuestras mediciones tuvimos un error menor al mayor posible.

En otras palabras, los errores en nuestras mediciones cayeron dentro del rango que la clase 2.5 toma en cuenta.
